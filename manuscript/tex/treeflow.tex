\section{Introduction}

Traditionally, phylogenetic analyses have been performed by specialist software. A number of software packages exist that implement a broad but predefined collection of models and associated specialized inference methods Typically, inference is handled by carefully crafted but computationally costly Markov Chain Monte Carlo (MCMC) methods. In contrast, in other realms of statistical analysis, \textit{probabilistic programming} software libraries have entered into widespread use. These allow the specification of almost any model as a probabilistic program, and inference is provided automatically with a generic inference method. Having the power of probabilistic programming in phylogenetic analyses could significantly accelerate research 

The structure of the phylogenetic tree object is a major barrier to implementing probabilistic programming for phylogenetics. It is not clear how the association between its discrete and continuous quantities (the topology and branch lengths respectively) should be represented and handled in inference. Also, the combinatorial explosion of the size of the phylogenetic state space presents a major challenge to any inference algorithm. Generic random search methods, as in the naive implementation of MCMC sampling, do not scale appropriately to allow inference on modern datasets with thousands of sequences.

